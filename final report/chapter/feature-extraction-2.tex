\newpage
\section{Speech Signal Processing}

\subsection{Analog-to-Digital Conversion}
Voices in real life are analog signals, hence before conducting digital signal processing techniques, analog-to-digital conversions are required.\\

Given a continuous-time signal $s(t)$, we define the \textit{sampled signal} by
\begin{equation}
s[n] = s(nT_s) = s(\frac{n}{F_s})
\end{equation}
where $T_s$ is the sampling interval and $F_s$ is the sampling rate.\\

$T_s$ should be carefully chosen in order to avoid distortion caused by aliasing. Telephony since the 1950s limits the information bandwidth to 300-3400 Hz \cite{EVW-report}. However, in normal conversational speech, the frequency content is mainly between 0-8000 Hz \cite{uysal2005bandwidth}. According to Nyquist-Shannon sampling theorem, we set the folding frequency $\frac{F_s}{2}$ = 8000 Hz, i.e. $F_s$ = 16 kHz.

%--------------------------------------------
%--------------------------------------------

\subsection{Pre-emphasis}

The speech production system inherently attenuates speech signal by a negative spectral slope per decade. In addition, human hearing is more sensitive to frequency band above 1 kHz \cite{picone1993signal}. However, as is depicted in Fig. \ref{zero_fft}, frequency components below 1 kHz predominantly comprise the spectrum. Hence, it is advantageous to employ a high-pass filter to amplify the high frequency range.

\begin{figure}[H]
\centering
\includegraphics[width=6in]{ang/pre_emphasis_filter}
\caption{Pre-emphasis Filters Comparison}
\label{pre_emphasis_filter}
\end{figure}

\begin{equation}
\label{high-pass-filter}
y[n] = x[n] - \alpha x[n-1]
\end{equation}

A 1st-order FIR filter represented by (\ref{high-pass-filter}) is widely implemented, including the previous group where $\alpha = 0.95$ \cite{EVW-report}. The merits of this FIR filter include simplicity and efficiency. However, Fig. \ref{pre_emphasis_filter} (red dash-dot line) shows that frequencies below 500 Hz are severely suppressed even though frequencies above 3 kHz are successfully amplified. Considering the potential interference caused by high-frequency noise, attenuating low frequencies too much will result in the decline of signal-to-noise ratio (SNR).

\begin{equation}
\label{shelving-filter}
y[n] = \frac{1}{a_0} \Big( b_0 x[n] + b_1 x[n-1] + b_2 x[n-3] - a_1 y[n-1] - a_2 y[n-2] \Big)
\end{equation}

Suggested by Professor Erik \textsc{Weyer}, we try to devise a shelving filter represented by (\ref{shelving-filter}) to pre-emphasize the speech signal. By trial and error, we eventually manage to obtain a appropriate filter that amplifies high frequency without attenuating low frequencies (shown in Fig. \ref{pre_emphasis_filter} by blue solid line), where

\begin{align}
\label{shleving-coef}
&\begin{cases}
a_0 = 1\\
a_1 = -1.523796\\
a_2 = 0.649345
\end{cases}
&\begin{cases}
b_0 = 1.861856\\
b_1 = -3.102851\\
b_2 = 1.366544
\end{cases}
\end{align}

Fig. \ref{zero_fft} shows the spectra of word `zero' before and after pre-emphasis filter.

\begin{figure}[H]
\centering
\includegraphics[width=\textwidth]{ang/zero_fft}
\caption{Word `zero' Spectral Analysis}
\label{zero_fft}
\end{figure}

%--------------------------------------------
%--------------------------------------------

\subsection{Framing \& Windowing}

Speech is time-varying signals. Due to the inertial motion of articulators (speech organs such as the tongue, lips and palate), speech can be considered statistically stationary in a short-time period (approximately 30 ms) \cite{brandstein1995practical}. The time period of 30 ms indicates 30 ms $\times$ 16000 Hz = 480. We take $N = 512$ samples per frame to achieve a power 2 for efficient Fast Fourier Transform.\\

The framing operation can be finished by multiplying the signal by a moving window. For the $j$-th frame and frame length $N$, mathematical equation is given in (\ref{eq:windowing}).

\begin{equation}
\label{eq:windowing}
s_j[n] =
\begin{cases}
w[n] s[n+jN] & N = 1, 2, \dots, N\\
0, & \text{otherwise}
\end{cases}
\end{equation}

The simplest and easiest-to-implement window is a rectangular window represented by (\ref{eq:rectagular-window}).
\begin{equation}
\label{eq:rectagular-window}
w[n] =
\begin{cases}
1, & N = 1, 2, \dots, N\\
0, & \text{otherwise}
\end{cases}
\end{equation}

However, the selection of a proper window always involves a trade-off between high \textbf{frequency resolution} and low \textbf{spectral leakage}. On the one hand, convolution with mainlobe smooths the estimate over nearby frequencies and the frequency resolution is determined by the width of the mainlobe. On the other hand, the sidelobes cause sidelobe energy to appear in the spectrum, i.e. spectral leakage. (from ELEN90058 \textit{Signal Processing} lecture slides by Erik \textsc{Weyer})

\begin{figure}[H]
\begin{minipage}[t]{0.5\linewidth}
\centering
\includegraphics[width=\textwidth]{ang/windows_time}
\caption{Windows in Time Domain}
\label{windows_time}
\end{minipage}
\begin{minipage}[t]{0.5\linewidth}
\centering
\includegraphics[width=\textwidth]{ang/windows_frequency}
\caption{Windows in Frequency Domain}
\label{windows_frequency}
\end{minipage}
\end{figure}

\begin{table}[H]
\begin{tabu} to \textwidth {XXXXX}
\toprule
Windows &Rectangular &Hanning &Hamming &Blackman\\
\hline
Mainlobe width &$0.125 \pi$ &$0.2353 \pi$ &$0.2985 \pi$ &$0.4 \pi$\\
\hline
Peak sidelobe &-13.2 dB &-31.5 dB &-39.8 dB &-58.6 dB\\
\bottomrule
\end{tabu}
\caption{Windows Properties for $N = 16$}
\label{windows_table}
\end{table}

In terms of the windows involved in Fig. \ref{windows_time}, Fig. \ref{windows_frequency} and Table \ref{windows_table}, \textit{rectangular} window has the best frequency resolution (narrowest mainlobe) at the expense of highest spectral leakage (biggest sidelode peak) while \textit{Blackman} window has the lowest spectral leakage (smallest sidelode peak) accompanied by worst frequency resolution (widest mainlobe). Eventually, we choose an intermediate \textit{Hamming} window given in (\ref{eq:hamming}).

\begin{equation}
\label{eq:hamming}
w[n] = \alpha - \beta \cos \bigg( \frac{2 \pi n}{N-1} \bigg) \quad n = 1, 2, \dots, N
\end{equation}
where $\alpha = 25/46 \approx 0.54$ and $\beta = 1 - \alpha \approx 0.46$.

\begin{figure}[H]
\begin{minipage}[t]{0.5\linewidth}
\centering
\includegraphics[width=\textwidth]{ang/hamming_bell_shape}
\caption{Information Loss}
\label{hamming_bell_shape}
\end{minipage}
\begin{minipage}[t]{0.5\linewidth}
\centering
\includegraphics[width=\textwidth]{ang/hamming_overlap}
\caption{Hamming with/without overlap}
\label{hamming_overlap}
\end{minipage}
\end{figure}

Fig. \ref{hamming_bell_shape} shows the effect of the framing operation instructed by (\ref{eq:rectagular-window}) and (\ref{eq:hamming}) for $N = 512$. The red ellipse in subplot 2 demonstrates the loss of information (data points near two borders are severely attenuated) due to the bell shape of the Hamming window shown in Fig. \ref{hamming_overlap} subplot 1.\\

In order to avoid information loss, we overlap each frame by half of the frame size. We choose to overlap a half primarily because the data points most attenuated in current frame will have largest gain in the next frame (shown by the cyan rectangle in Fig. \ref{hamming_overlap}). Thus, information will be effectively preserved.

%--------------------------------------------
%--------------------------------------------

\subsection{Threshold}

%--------------------------------------------
%--------------------------------------------

\subsection{Mel Filter Bank Processing}

\subsubsection{Power Spectrum}

%--------------------------------------------

\subsubsection{Bank Filtering}

\begin{figure}[H]
\centering
\includegraphics[width=6in]{ang/mel_filter_bank_gain}
\caption{Mel Filter Bank Gain}
\end{figure}

\begin{figure}[H]
\centering
\includegraphics[width=6in]{ang/mel_bank_9}
\caption{Bank Filtering Demonstration}
\end{figure}

%--------------------------------------------

\subsubsection{Log Scaling}

%--------------------------------------------

\subsubsection{Discrete Cosine Transform}
