\chapter{Conclusion}
\label{chapter:conclusion}

Voice-controlled television has been successfully implemented on ADSP-BF548 by integrating the word recognition, adaptive noise cancellation and home automation techniques.\\

Theoretical study of noise reduction schemes has been conducted, normalized least mean square algorithm was adopted synthesizing the project scenario and the restrictions of the DSP board. Besides, research has been undertaken in speech features and speech signal processing. Choices we made such as the window type (Hamming) and the number of MFCCs ($F = 13$) are explained and justified. There also have been detailed discussions related to HMM and speech modeling. Optimal parametric choices (such as the number of states) have been found. With the aid of new training techniques, more accurate models are trained and exported, thus recognition rate of confusable words are dramatically increased.

%----------------------------------------------------------------------------------------
%	Section
%----------------------------------------------------------------------------------------

\section{Improvements on previous work}

Working on a follow-up project, we are able to draw valuable experience from previous groups while we have also achieved some new accomplishments.

\begin{enumerate}
\item Adaptive noise cancellation scheme is firstly introduced. Least Mean Square algorithm makes recognition system more resilient to accessible noise.
\item Cross-validation technique is utilized to make full use of the limited speech corpus by partitioning a sample of data into complementary subsets, analyzing one subset and validating by the other subset.
\item In addition to Baum-Welch algorithm, Maximum Mutual Information algorithm is brought in. Thereby, relationship between confusable models is established.
\item More computations are conducted in fixed-point arithmetic and reusable coefficients are precomputed during implementation on DSP. Thus, program efficiency and memory management are further optimized.
\end{enumerate}

%----------------------------------------------------------------------------------------
%	Section
%----------------------------------------------------------------------------------------

\section{Future Improvements}

\subsection{Adaptive Noise Cancellation}

The design and implementation of the current system support accessible noise reduction, i.e. the pure noise can be directly imported from the TV \texttt{Line Out}. Wiring will be tedious if more multi-media devices such as personal computer and audio speaker are taken into account. Besides, noise cancellation mechanism has better performance when dealing with speech noise than coping with music because the sampling frequency (16 kHz) is not high enough. However, we cannot increase the sampling frequency disparately due to the computational limitation of ADSP-BF548. In fact, in order to satisfy the real-time processing specification, we update the weights in LMS algorithm every four data points. Implementing fixed-point arithmetic or migrating to floating-point processor such as SHARC\textsuperscript{\textregistered} are potential solutions.

%--------------------------------------------

\subsubsection{Inaccessible Noise Cancellation}

Inaccessible noise refers to the signal that cannot be conveniently obtained, such as the noise from the electric fan or outside the window.\\

Inspired by the voice-controlled speaker `Echo' with an array of seven microphones, we start to look into noise reduction and speech enhancement techniques based on microphones array. By exploiting beam forming technology, `Echo' can receive the speech command from across the room even while music is playing. This scenario resembles the working condition of our system. \cite{mccowan2003microphone} and \cite{spalt2011background} have shown the feasibility of this scheme.

%--------------------------------------------
%--------------------------------------------

\subsection{Support of More Home Appliances}
In order to control more home appliances, we firstly need to extend the number of words supported by our system. Besides, more home appliances will be integrated into the system. For example, air-conditioners can be controlled via infrared emitter and lights can be switched by a relay. With further development and investment, our word recognition system has the potential to become the hub of home automation.
