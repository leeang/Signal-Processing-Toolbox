\section{Speech Signal Processing}

\begin{frame}
\frametitle{Analog-to-Digital Conversion}
\begin{itemize}
\item Telephony since the 1950s limits the information bandwidth to 300-3400 Hz.
\item In normal conversational speech, the frequency content is mainly between 0-8000 Hz \cite{uysal2005bandwidth}.
\item We choose sampling rate $F_s$ = $2 \times 8000$ Hz = 16000 Hz.\\(Nyquist-Shannon sampling theorem)
\end{itemize}
\end{frame}

%----------------------------------------------------------------------------------------
%	Subsection
%----------------------------------------------------------------------------------------

\begin{frame}
\frametitle{Pre-emphasis}
\begin{itemize}
\item Pre-emphasis filter is essentially an \textbf{high-pass} filter.
\vspace{15pt}
\item Voiced speech naturally have an attenuation of $\sim$ 20 dB per decade due to physiological characteristics of the speech production system \cite{picone1993signal}.
\begin{itemize}
\item compensates this natural attenuation before spectral analysis.
\end{itemize}
\vspace{15pt}
\item Hearing is more sensitive to the components higher than 1 kHz.
\begin{itemize}
\item caters to human perception of sound
\end{itemize}
\end{itemize}
\end{frame}

%----------------------------------------------------------------------------------------
%	Subsection
%----------------------------------------------------------------------------------------

\begin{frame}
\frametitle{Framing \& Windowing}
\begin{itemize}
	\item speech signals are time-varying signals.
	\item the inertial motion of articulators
	\begin{itemize}
		\item speech can be considered statistically stationary in a short-time period ($\sim$ 30 ms) \cite{brandstein1995practical}.
		\item 30 ms $\longrightarrow$ 480 samples $\longrightarrow$ 512 (Radix-2 FFT)
	\end{itemize}
	\item Hamming window
	\begin{itemize}
		\item a compromise between \textbf{frequency resolution} and \textbf{spectral leakage}
	\end{itemize}
\end{itemize}
\end{frame}

%----------------------------------------------------------------------------------------
%	Subsection
%----------------------------------------------------------------------------------------

\begin{frame}
\frametitle{Threshold}
\begin{itemize}
\item distinguish informative frames from silent frames
\item two metrics: frame \textit{energy} and frame \textit{zero-crossing count}
\end{itemize}

\begin{table}[H]
\centering
\caption{properties of different frame types}
\begin{tabu} to 0.9\textwidth {X[c]X[c]X[c]}
\toprule
Type &Energy &Zero-crossing count\\
\hline
Voiced &high &low\\
\hline
Unvoiced &low &high\\
\hline
Silent &low &low\\
\bottomrule
\end{tabu}
\end{table}
\end{frame}

%--------------------------------------------
%--------------------------------------------

\begin{frame}
\begin{figure}[H]
\centering
\includegraphics[width=\textwidth]{ang/threshold2}
\caption{decision-making strategy}
\end{figure}
\end{frame}

%----------------------------------------------------------------------------------------
%	Subsection
%----------------------------------------------------------------------------------------

\begin{frame}
\frametitle{Mel-frequency Cepstral Coefficients Extraction}
\begin{equation}
X[n] = \mathcal{F}^{-1} \left\{\log_{10} \left( |\mathcal{F}\{s[n]\}|^2 \right) \right\}
\end{equation}
\vspace{10pt}

The whole process to evaluate \textit{power cepstrum} can be divided into three procedures.
\begin{enumerate}
\item Compute the Discrete Fourier Transform $S_j[k]$ and corresponding power spectrum $\hat{S}_j[k]$ of a time-domain signal $s_j[n]$.
\item Take the logarithm of the power spectrum $\hat{S}_j[k]$.
\item Conduct inverse Fourier transform.
\end{enumerate}
\end{frame}

%--------------------------------------------
%--------------------------------------------

\begin{frame}
\begin{itemize}
	\item Human hearing responds to the entire critical band instead of individual frequencies in this band.
	\begin{itemize}
		\item MFCC calculates the total power within each certain mel-scale band prior to log scaling.
	\end{itemize}
	\item MFCC substitutes Discrete Cosine Transform for inverse Fourier transform to reduce the computational complexity.
\end{itemize}
\vspace{10pt}

MFCC extraction process
\begin{enumerate}
\item Power spectrum
\item Bank filtering
\item Log scaling
\item Discrete cosine transform
\end{enumerate}
\end{frame}

%--------------------------------------------
%--------------------------------------------

\begin{frame}
\frametitle{I. Power spectrum}

Discrete Fourier Transform constitutes the cornerstone of spectrum analysis.
\begin{equation}
S_j[k] = \sum_{n=1}^{N} s_j[n] W_N^{(n-1) k} \quad k = 1, 2, \dots, N
\end{equation}
\begin{equation}
W_N = e^{\frac{- 2\pi i}{N}}
\end{equation}

\begin{itemize}
	\item The computations required by DFT increase dramatically as length $N$ increases.
	\begin{itemize}
		\item Infeasible to directly implement on large sequences.
	\end{itemize}
	\item Fast Fourier Transform makes implementation of DFT practical in real-time processing.
	\begin{itemize}
		\item Radix-2 algorithm restricts the length of input sequence to $2^n$.
	\end{itemize}
\end{itemize}
\end{frame}

%--------------------------------------------

\begin{frame}
The DFT $X[k]$ of a real sequence $x[n] \in \mathbb{R}$ is a conjugate symmetric sequence (from ELEN90058 \textit{Signal Processing} Workshop 3).
\begin{equation}
X[k] = X^*[\langle-k\rangle_{N}] = X^*[N-k]
\end{equation}

Last $(\frac{N}{2} - 1)$ points can be discarded when computing the power spectrum.
\begin{equation}
\hat{S}_j[k] = |S_j(k)|^2 \quad k = 1, 2, \dots, \frac{N}{2} + 1
\end{equation}
\end{frame}

%--------------------------------------------
%--------------------------------------------

\begin{frame}
\frametitle{II. Bank Filtering}
\begin{figure}[H]
\centering
\includegraphics[width=0.9\textwidth, trim={0 0.6cm 0 0.6cm}, clip]{ang/mel_triangle}
\end{figure}
\end{frame}

%--------------------------------------------

\begin{frame}
\begin{enumerate}
\item Each power spectrum data point $\hat{S}_j[k]$ (\textcolor{gold_matlab}{gold circle $\circ$}) is multiplied by the corresponding gain $H_{mel}[m, k]|_{m=9}$ (\textcolor{orange_matlab}{orange asterisk $*$}).
\item \textcolor{orange_matlab}{Orange crosses $\times$} represent the filtered power spectrum.
\item Total power within bank $m=9$ is the sum of all filtered power data points.
\end{enumerate}

\begin{figure}[H]
\centering
\includegraphics[width=3in, trim={0 0.6cm 0 0.6cm}, clip]{ang/bank_filter_demostration}
\end{figure}
\end{frame}

%--------------------------------------------
%--------------------------------------------

\begin{frame}
\frametitle{III. Log Scaling}
\begin{equation}
\hat{X}_j[m] = \log_{10}(X_j[m]) \quad m = 1, 2, \dots, M
\end{equation}

\begin{enumerate}
\item Log scaling makes the system more resilient to both very quiet and very loud sound.
\item Log scale in dB imitates human nonlinear perception to loudness \cite{farin2008mathematical}.
\item Without taking logarithm, recognition accuracy is severely reduced \cite{tan2008automatic}.
\end{enumerate}
\end{frame}

%--------------------------------------------
%--------------------------------------------

\begin{frame}
\frametitle{IV. Discrete Cosine Transform}
IDFT is replaced by discrete cosine transform (DCT) due to the symmetric and real characteristic of log power spectrum $\hat{X}_j[m]$ \cite{picone1993signal, iser2008bandwidth}.
\begin{equation}
\hat{C}_j[n] = \sqrt{\frac{2}{M}} \sum^{M}_{m=1} \hat{X}_j[m] \cos \left( \frac{\pi}{M} (m - 0.5) (n-1) \right) \quad n = 1, 2, \dots, F
\end{equation}

\begin{itemize}
	\item The order of DCT ($F$) determines the amount of MFCCs. \cite{tan2008automatic}
	\begin{itemize}
		\item Higher-order coefficients incorporates excitation information.
		\item Lower-order coefficients indicate the slowly varying vocal tract. (more useful for speech recognition)
	\end{itemize}
	\item European Telecommunications Standards Institute adopts $F = 13$ in their speech recognition standard \cite{etsi2001202}.
	\item We also condense the $M$-point sequence $\hat{X}_j[m]$ into a shorter $F$-point sequence $\hat{C}_j[n]$. The choice of $F$ will be further discussed in \textit{Design \& Performance} section.
\end{itemize}
\end{frame}
