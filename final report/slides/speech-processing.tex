\section{Speech Signal Processing}

\begin{frame}
\frametitle{Pre-emphasis}
\begin{itemize}
\item Pre-emphasis filter is essentially an \textbf{high-pass} filter.
\vspace{15pt}
\item Voiced speech has a natural attenuation of $\sim$ 20 dB per decade.
\begin{itemize}
\item physiological characteristics of the speech production system \cite{picone1993signal}.
\item Pre-emphasis compensates this natural attenuation.
\end{itemize}
\vspace{15pt}
\item Hearing is more sensitive to the components higher than 1 kHz.
\begin{itemize}
\item Pre-emphasis caters to human perception of sound.
\end{itemize}
\end{itemize}
\end{frame}

%----------------------------------------------------------------------------------------
%	Subsection
%----------------------------------------------------------------------------------------

\begin{frame}
\frametitle{Framing \& Windowing}
\begin{itemize}
	\item speech signals are time-varying signals.
	\item speech can be considered stationary in a short-time period ($\sim$ 30 ms)
	\begin{itemize}
		\item the inertial motion of articulators \cite{brandstein1995practical}.
		\item 30 ms $\longrightarrow$ 480 samples $\longrightarrow$ 512 (Radix-2 FFT)
	\end{itemize}
	\item Hamming window
	\begin{itemize}
		\item a compromise between \textbf{frequency resolution} and \textbf{spectral leakage}
	\end{itemize}
\end{itemize}
\end{frame}

%----------------------------------------------------------------------------------------
%	Subsection
%----------------------------------------------------------------------------------------

\begin{frame}
\frametitle{Threshold}
\begin{itemize}
\item distinguish informative frames from silent frames
\item two metrics: frame \textit{energy} and frame \textit{zero-crossing count}
\end{itemize}

\begin{figure}[H]
\centering
\includegraphics[width=\textwidth]{ang/threshold2}
\caption{decision-making strategy}
\end{figure}
\end{frame}

%----------------------------------------------------------------------------------------
%	Subsection
%----------------------------------------------------------------------------------------

\begin{frame}
\frametitle{Mel-frequency Cepstral Coefficients Extraction}

Power cepstrum
\begin{equation}
X[n] = \mathcal{F}^{-1} \left\{\log_{10} \left( |\mathcal{F}\{s[n]\}|^2 \right) \right\}
\end{equation}
\vspace{10pt}

\begin{enumerate}
\item Power spectrum based on Discrete Fourier Transform.
\item Log scaling.
\item Inverse Fourier Transform.
\end{enumerate}
\end{frame}

%--------------------------------------------
%--------------------------------------------

\begin{frame}
\frametitle{I. Power spectrum}

Discrete Fourier Transform
\begin{equation}
S_j[k] = \sum_{n=1}^{N} s_j[n] W_N^{(n-1) k} \quad k = 1, 2, \dots, N
\end{equation}
\begin{equation}
W_N = e^{\frac{- 2\pi i}{N}}
\end{equation}

Power Spectrum
\begin{equation}
\hat{S}_j[k] = |S_j[k]|^2 \quad k = 1, 2, \dots, \frac{N}{2} + 1
\end{equation}
\end{frame}

%--------------------------------------------
%--------------------------------------------

\begin{frame}
\frametitle{II. Bank Filtering}
\begin{itemize}
	\item Human hearing responds to the entire critical band instead of individual frequencies in this band.
	\item $M = 20$ banks
\end{itemize}

\begin{figure}[H]
\centering
\includegraphics[width=0.9\textwidth, trim={0 0.6cm 0 0.6cm}, clip]{ang/mel_triangle}
\end{figure}
\end{frame}

%--------------------------------------------
%--------------------------------------------

\begin{frame}
\frametitle{III. Log Scaling}
\begin{equation}
\hat{X}_j[m] = \log_{10}(X_j[m]) \quad m = 1, 2, \dots, M
\end{equation}

\begin{enumerate}
\item Log scaling makes the system more resilient to both very quiet and very loud sound.
\item Log scale in dB imitates human nonlinear perception to loudness \cite{farin2008mathematical}.
\item Without taking logarithm, recognition accuracy is severely reduced \cite{tan2008automatic}.
\end{enumerate}
\end{frame}

%--------------------------------------------
%--------------------------------------------

\begin{frame}
\frametitle{IV. Discrete Cosine Transform}
IDFT is replaced by Discrete Cosine Transform due to the symmetric and real characteristic of log power spectrum $\hat{X}_j[m]$ \cite{picone1993signal, iser2008bandwidth}.

\begin{equation}
\hat{C}_j[n] = \sqrt{\frac{2}{M}} \sum^{M}_{m=1} \hat{X}_j[m] \cos \left( \frac{\pi}{M} (m - 0.5) (n-1) \right) \quad n = 1, 2, \dots, F
\end{equation}

\begin{itemize}
	\item Condense $M = 20$ points into $F = 13$ points. \cite{tan2008automatic}
	\begin{itemize}
		\item Higher-order coefficients incorporate excitation information.
		\item Lower-order coefficients indicate the slowly varying vocal tract.
	\end{itemize}
\end{itemize}
\end{frame}
