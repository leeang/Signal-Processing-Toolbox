\section{Final Implementation}

%----------------------------------------------------------------------------------------
%	Subsection
%----------------------------------------------------------------------------------------

\begin{frame}
\frametitle{Improvements on previous work (word recognition)}
\begin{enumerate}
\item Coefficients that intensively require floating-point computations are precomputed. (Hamming window, Mel-bank filter gain and discrete cosine transform)
\item Sparse matrices are stored in ingeniously designed structures. Meaningless multiplications with zeros are illuminated. (Mel bank filter gain, inversed variance matrices and transition matrices)
\item Symmetry is sufficiently taken in to consideration. (FFT of real sequence and discrete cosine transform)
\end{enumerate}

\begin{itemize}
\item Hamming window: $1 - \frac{0.014552 \text{ ms}}{1.012452 \text{ ms}} = 98.6\%$
\item DCT: $1 - \frac{0.006482 \text{ ms}}{0.287135 \text{ ms}} = 97.7\%$
\end{itemize}
\end{frame}

%--------------------------------------------

\begin{frame}
Feature Extraction \& Recognition
\begin{itemize}
\item The average processing time of these 23 datasets is 95.8 ms.
\item The maximal processing time is 150 ms.
\end{itemize}

\begin{figure}[H]
\centering
\includegraphics[width=3in, trim={0 0.6cm 0 0.6cm}, clip]{ang/processing_time}
\end{figure}
\end{frame}

%----------------------------------------------------------------------------------------
%	Subsection
%----------------------------------------------------------------------------------------

\begin{frame}
\frametitle{Performance on DSP Board}
We randomly test the word recognition system on DSP and compare the outcomes computed by DSP and MATLAB for 23 times.
\begin{itemize}
	\item for dataset $n$ ($n = 1, 2, \dots, 23$) and word $k$ ($k = 1, 2, \dots, 27$)
	\item $p_{MATLAB}[n, k]$ denotes the probability (natural-log scale) computed by MATLAB
	\item $p_{DSP}[n, k]$ denotes the probability (natural-log scale) computed by DSP
\end{itemize}

The \textit{relative error} of word $k$ in dataset $n$
\begin{equation}
\varepsilon[n, k] = \frac{p_{DSP}[n, k] - p_{MATLAB}[n, k]}{p_{MATLAB}[n, k]}
\end{equation}
The \textit{Root Mean Square Error} of dataset $n$
\begin{equation}
\epsilon_{RMSE}[n] = \sqrt{\frac{1}{27} \sum_{k = 1}^{27} (\varepsilon[n, k])^2}
\end{equation}
\end{frame}

%--------------------------------------------
%--------------------------------------------

\begin{frame}
Absolute relative error $|\varepsilon[n, k]|$ ranges from $1 \times 10^{-6}$ to $9 \times 10^{-3}$ and dataset 1 has the largest deviation (\textcolor{navy_matlab}{navy solid line}).

\begin{figure}[H]
\centering
\includegraphics[width=3in, trim={0 0.6cm 0 0.6cm}, clip]{ang/relative_error}
\end{figure}
\end{frame}

%--------------------------------------------
%--------------------------------------------

\begin{frame}
Dataset 1 has the largest root mean square error $\epsilon_{RMSE}[n]$ consistently.

\begin{figure}[H]
\centering
\includegraphics[width=3in, trim={0 0.6cm 0 0.6cm}, clip]{ang/root_mean_square_error}
\caption{the root mean square error $\epsilon_{RMSE}[n]$ of each dataset $n$}
\end{figure}
\end{frame}

%--------------------------------------------
%--------------------------------------------

\begin{frame}
\begin{itemize}
\item The computation error has no influence on recognition result.
\item Take the worst case (dataset 1) as an example.
\item Probabilities of 27 words computed by MATLAB (\textcolor{navy_matlab}{navy circle $\circ$}) and DSP (\textcolor{orange_matlab}{orange cross $\times$}) are so close to each other. (much less than the differences between probabilities.)
\end{itemize}

\begin{figure}[H]
\centering
\includegraphics[width=3in, trim={0 0.6cm 0 0.6cm}, clip]{ang/word_probabilities}
\end{figure}
\end{frame}

%----------------------------------------------------------------------------------------
%	Subsection
%----------------------------------------------------------------------------------------

\begin{frame}
\frametitle{Final system implemented on DSP board}
\begin{itemize}
	\item Adaptive noise cancellation based on Least Mean Square Algorithm
	\item Real-time recognition of spoken words
	\item Result indication via a five-LED array
		\begin{itemize}
		\item \LED\offLED\offLED\offLED\offLED\onLED $\longrightarrow$ $(00001)_2 = 1$ $\longrightarrow$ word \textit{one}
		\item \LED\offLED\onLED\onLED\onLED\onLED $\longrightarrow$ $(01111)_2 = 15$ $\longrightarrow$ word \textit{fifteen}
		\item \LED\onLED\offLED\onLED\offLED\onLED $\longrightarrow$ $(10101)_2 = 21$ $\longrightarrow$ word \textit{zero}
		\item \LED\offLED\offLED\offLED\offLED\offLED $\longrightarrow$ $(00000)_2 = 0$ $\longrightarrow$ error
		\end{itemize}
	\item YouTube controller
\end{itemize}
\end{frame}

%----------------------------------------------------------------------------------------
%	Subsection
%----------------------------------------------------------------------------------------

\begin{frame}
\frametitle{Utilization in Home Automation}
\begin{itemize}
	\item Arduino\textsuperscript{\textregistered} board as an intermediate
		\begin{itemize}
		\item the open-source ecosystem of abundant resources already available to Arduino developers
		\item infrared emitter $\longrightarrow$ remote control of TV
		\item WIFI module $\longrightarrow$ wireless communication
		\item relay $\longrightarrow$ AC light switch
		\end{itemize}
\end{itemize}

\begin{figure}[H]
\centering
\includegraphics[width=0.5\textwidth]{ang/relay-arduino}
\caption{Control AC Light using Arduino with Relay Module \cite{relay-arduino}}
\end{figure}
\end{frame}

\begin{frame}
Control YouTube on PC
\begin{enumerate}
	\item A python program keeps listening the serial port (wired connection) or the \texttt{Telnet} port (wireless connection).
	\item The python program simulates a keyboard press according to the received command.
	\item YouTube on PC becomes controlled by speech command.
\end{enumerate}
\end{frame}
